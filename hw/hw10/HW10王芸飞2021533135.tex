\documentclass{article}

\usepackage{fancyhdr}
\usepackage{extramarks}
\usepackage{amsmath}
\usepackage{amsthm}
\usepackage{amsfonts}
\usepackage{tikz}
\usepackage[plain]{algorithm}
\usepackage{algpseudocode}
\usepackage{enumerate}
\usepackage{tikz}
\usepackage{xifthen}
\usepackage{xparse}
\usepackage{amsmath, amssymb}
\usepackage{lipsum}
\usetikzlibrary{automata,positioning}

%
% Basic Document Settings
%  

\topmargin=-0.45in
\evensidemargin=0in
\oddsidemargin=0in
\textwidth=6.5in
\textheight=9.0in
\headsep=0.25in

\linespread{1.1}

\pagestyle{fancy}
\lhead{\hmwkAuthorName}
\chead{\hmwkClass : \hmwkTitle}
\rhead{\firstxmark}
\lfoot{\lastxmark}
\cfoot{\thepage}

\renewcommand\headrulewidth{0.4pt}
\renewcommand\footrulewidth{0.4pt}

\setlength\parindent{0pt}

%
% Create Problem Sections
%

\newcommand{\enterProblemHeader}[1]{
    \nobreak\extramarks{}{Problem \arabic{#1} continued on next page\ldots}\nobreak{}
    \nobreak\extramarks{Problem \arabic{#1} (continued)}{Problem \arabic{#1} continued on next page\ldots}\nobreak{}
}

\newcommand{\exitProblemHeader}[1]{
    \nobreak\extramarks{Problem \arabic{#1} (continued)}{Problem \arabic{#1} continued on next page\ldots}\nobreak{}
    \stepcounter{#1}
    \nobreak\extramarks{Problem \arabic{#1}}{}\nobreak{}
}

\newcommand*\circled[1]{\tikz[baseline=(char.base)]{
		\node[shape=circle,draw,inner sep=2pt] (char) {#1};}}


\setcounter{secnumdepth}{0}
\newcounter{partCounter}
\newcounter{homeworkProblemCounter}
\setcounter{homeworkProblemCounter}{1}
\nobreak\extramarks{Problem \arabic{homeworkProblemCounter}}{}\nobreak{}

%
% Homework Problem Environment
%
% This environment takes an optional argument. When given, it will adjust the
% problem counter. This is useful for when the problems given for your
% assignment aren't sequential. See the last 3 problems of this template for an
% example.
%

\newenvironment{homeworkProblem}[1][-1]{
    \ifnum#1>0
        \setcounter{homeworkProblemCounter}{#1}
    \fi
    \section{Problem \arabic{homeworkProblemCounter}}
    \setcounter{partCounter}{1}
    \enterProblemHeader{homeworkProblemCounter}
}{
    \exitProblemHeader{homeworkProblemCounter}
}

%
% Homework Details
%   - Title
%   - Class
%   - Due date
%   - Name
%   - Student ID

\newcommand{\hmwkTitle}{Homework\ \#10}
\newcommand{\hmwkClass}{Probability \& Statistics for EECS}
\newcommand{\hmwkDueDate}{April 23, 2023}
\newcommand{\hmwkAuthorName}{Wang Yunfei}
\newcommand{\hmwkAuthorID}{2021533135}


%
% Title Page
%

\title{
    \vspace{2in}
    \textmd{\textbf{\hmwkClass:\\  \hmwkTitle}}\\
    \normalsize\vspace{0.1in}\small{Due\ on\ \hmwkDueDate\ at 23:59}\\
	\vspace{4in}
}

\author{
	Name: \textbf{\hmwkAuthorName} \\
	Student ID: \hmwkAuthorID}
\date{}

\renewcommand{\part}[1]{\textbf{\large Part \Alph{partCounter}}\stepcounter{partCounter}\\}

%
% Various Helper Commands
%

% Useful for algorithms
\newcommand{\alg}[1]{\textsc{\bfseries \footnotesize #1}}
% For derivatives
\newcommand{\deriv}[1]{\frac{\mathrm{d}}{\mathrm{d}x} (#1)}
% For partial derivatives
\newcommand{\pderiv}[2]{\frac{\partial}{\partial #1} (#2)}
% Integral dx
\newcommand{\dx}{\mathrm{d}x}
% Alias for the Solution section header
\newcommand{\solution}{\textbf{\large Solution}}
% Probability commands: Expectation, Variance, Covariance, Bias
\newcommand{\E}{\mathrm{E}}
\newcommand{\Var}{\mathrm{Var}}
\newcommand{\Cov}{\mathrm{Cov}}
\newcommand{\Bias}{\mathrm{Bias}}

\begin{document}

\maketitle

\pagebreak

\begin{homeworkProblem}[1]
\solution
\begin{enumerate}[(a)]
    \item
    X,Y are discrete
    \begin{align*}
        P(X=x)&=P(\cup_y(X=x,Y=y ) )\\
                &=\sum_{y}P(X=x,Y=y)\\
        Because\ of\ bayes\ rule\\
                &=\sum_{y}P(X=x|Y=y)P(Y=y)
    \end{align*}
    This uses Bayes rule so the confine of it is $P(Y=y)>0$.
\end{enumerate}
\begin{enumerate}[(b)]
    \item
    X is discrete, Y is continuous, so from the bayes rule we have,
    \begin{align*}
        f_Y(y|X=x)&=\frac{P(X=x|Y=y)f_Y(y)}{P(X=x)}\\
            P(X=x|Y=y)f_Y(y)&=f_Y(y|X=x)P(X=x)\\
            \int_{-\infty}^{\infty} P(X=x|Y=y)f_Y(y) \,dy&=\int_{-\infty}^{\infty} f_Y(y|X=x)P(X=x) \,dy\\ 
            &=P(X=x)\int_{-\infty}^{\infty} f_Y(y|X=x) \,dy\\
            Because\ of\ the\ property\ of\ the\ conditional\ Probability\\
            &=P(X=x)*1\\
            &=P(X=x)\\
    \end{align*}
    Then we get it, $P(X=x)=\int_{-\infty}^{\infty} P(X=x|Y=y)f_Y(y) \,dy$\\
   
\end{enumerate}
\begin{enumerate}[(c)]
    \item
    X is continuous, Y is discrete, so from the discrete form of LOTP, we can have, 
    \begin{align*}
    P(X\in (x-\varepsilon ,x+\varepsilon ))&=\sum_{y}^{}P(X\in (x-\varepsilon ,x+\varepsilon )|Y=y)P(Y=y)\\
    lim_{\varepsilon \rightarrow 0}\frac{P(X\in (x-\varepsilon ,x+\varepsilon ))}{2\varepsilon}&=lim_{\varepsilon \rightarrow 0}\sum_{y}^{}\frac{P(X\in (x-\varepsilon ,x+\varepsilon )|Y=y)}{2\varepsilon} P(Y=y)\\
    Because\ the\ definition\ of\ integral\ and\ when\ \varepsilon\ approaches\ to\ 0 \\
    f_X(x)&=\sum_{y}^{}f_X(x|Y=y)P(Y=y)
    \end{align*}
    Then we get it.
\end{enumerate}
\begin{enumerate}[(d)]
    \item
    X,Y are continuous,so from the continuous form of Bayes rule, we can have, 
    \begin{align*}
    f_{Y|X}(y|x)&=\frac{f_{X|Y}(x|y)f_Y(y)}{f_X(x)}\\
    f_{Y|X}(y|x)f_X(x)&=f_{X|Y}(x|y)f_Y(y)\\
    \int_{-\infty}^{\infty} f_{Y|X}(y|x)f_X(x) \,dy&=\int_{-\infty}^{\infty} f_{X|Y}(x|y)f_Y(y) \,dy\\ 
    f_X(x)\int_{-\infty}^{\infty} f_{Y|X}(y|x)\,dy&=\int_{-\infty}^{\infty} f_{X|Y}(x|y)f_Y(y) \,dy\\ 
    Because\ of\ the\ property\ of\ the\ conditional\ Probability\\
    f_X(x)*1&=\int_{-\infty}^{\infty} f_{X|Y}(x|y)f_Y(y) \,dy\\ 
    f_X(x)&=\int_{-\infty}^{\infty} f_{X|Y}(x|y)f_Y(y) \,dy
    \end{align*}
    Then we get it.
\end{enumerate}
\end{homeworkProblem}
\newpage
\begin{homeworkProblem}[2]

\begin{enumerate}[(a)]
    \item
   assume U is the time of arrival of next Blissville company bus, $U\sim Unif(0,15)$.\\
   assume E is the time of arrival of next Blotchiville company bus , $E\sim Expo(\frac{1}{15})$.\\
   Then from LOTP we can have,
   \begin{align*}
    P(E<U)&=\int_{0}^{15} P(E<U|U=u)f_U(u) \,du \\
    Because\ of\ independent\\
    &=\frac{1}{15}\int_{0}^{15} P(E<U) \,du \\
    &=\frac{1}{15}\int_{0}^{15} 1-e^{-\frac{1}{15}u}\,du \\
    &=\frac{1}{e}
   \end{align*}
\end{enumerate}
\begin{enumerate}[(b)]
    \item
    Let X is the time of $min(U,E)$.\\
    Then $P(X>x)=P(U>x,E>x)=P(U>x)P(E>x)=\frac{15-x}{15}*(1-1+e^{-\frac{1}{15}x})=\frac{15-x}{15}e^{-\frac{1}{15}x}$.\\
    Then $F_X(x)=1-p(X>t)=1-\frac{15-x}{15}e^{-\frac{1}{15}x}$, for $0<=x<=15$ and otherwise =0.\\
\end{enumerate}

\end{homeworkProblem}
\newpage
\begin{homeworkProblem}[3]
\solution
\begin{enumerate}[(a)]
    \item
    When $x+y\neq n$, $P(X=x,Y=y,N=n)=0$\\
    When $x+y=n$, 
    \begin{align*}
        P(X=x,Y=y,N=n)&=P(X=x,Y=y,N=x+y)\\
        &=P(X=x,Y=y|N=x+y)P(N=x+y)\\
        &=P(X=x,Y=n-x)P(N=x+y)\\
        &=\binom{n}{x}p^x(1-p)^y\frac{\lambda^ne^{-\lambda}}{n!}
    \end{align*}
   They are not independent, because when $x+y\neq n$, $P(X=x,Y=y,N=n)=0$, but at the same time $P(X=x)P(Y=y)P(N=n)\neq 0$.
\end{enumerate}
\begin{enumerate}[(b)]
    \item
when $x>n$, $P(X=x,N=n)=0$\\
when $x<=n$,
\begin{align*}
    P(X=x,N=n)&=P(X=x|N=n)P(N=n)\\
    &=\binom{n}{x}p^x(1-p)^{n-x}\frac{\lambda^ne^{-\lambda}}{n!}
\end{align*}
Then we can consider $P(X=x)$ from LOTP,
\begin{align*}
    P(X=x)&=\sum_{n=0}^{\infty}P(X=x|N=n)P(N=n)\\
    &=\sum_{n=x}^{\infty}\binom{n}{x}p^x(1-p)^{n-x}\frac{\lambda^ne^{-\lambda}}{n!}\\
    Let\ n-x=t\\
    &=\frac{(\lambda p)^xe^{-\lambda}}{x!}\sum_{t=0}^{\infty}\frac{[\lambda(1-p)]^t}{t!}\\
    From\ talor\ expansion\\
    &=\frac{(\lambda p)^xe^{-\lambda}}{x!}e^{\lambda (1-p)}\\
    &=\frac{(\lambda p)^xe^{-\lambda p}}{x!}
\end{align*}
Therefore $P(X=x|N=n)\neq P(X=x)$ obviously. Then $N,X$ are dependent.
\end{enumerate}
\begin{enumerate}[(c)]
    \item
\begin{align*}
    P(X=x,Y=y)&=\sum_{n=0}^{\infty}P(X=x,Y=y,N=n)\\
    From\ the\ result\ of\ (a),\ n=x+y,\ otherwise\ P=0\\
    &=\binom{n}{x}p^x(1-p)^y\frac{\lambda^ne^{-\lambda}}{n!}\\
\end{align*}
Like (b), we can also analyse $P(Y=y)$,
\begin{align*}
    P(Y=y)=\sum_{n=0}^{\infty}P(Y=y|N=n)P(N=n)\\
    &=\sum_{n=0}^{\infty}\binom{n}{y}p^{n-y}(1-p)^{y}\frac{\lambda^ne^{-\lambda}}{n!}\\
    Let\ n-y=t\ and\ from\ talor\ expansion\\
    &=\frac{(\lambda (1-p))^ye^{-\lambda (1-p)}}{y!}
\end{align*}
Therefore $P(X=x)P(Y=y)=\frac{(\lambda (1-p))^ye^{-\lambda (1-p)}}{y!}*\frac{(\lambda p)^xe^{-\lambda p}}{x!}$.\\
Then we can get $P(X=x,Y=y)=P(X=x)P(Y=y)$, and then they are independent.
\end{enumerate}
\begin{enumerate}[(d)]
    \item
From the work we have done, we can know $X\sim Pois(\lambda p)$, $Y\sim Pois(\lambda (1-p))$ and they are independent.\\
    \begin{align*}
    Cov(N,X)&=Cov(X+Y,X)\\
    &=Cov(X,X)+Cov(X,Y)\\
    From\ the\ property\ of\ Covariance\\
    &=Var(X)\\
    &=\lambda p\\
    Corr(N,X)&=\frac{Cov(N,X)}{\sqrt[2]{\lambda \lambda p}}\\
    &=\sqrt{p}
\end{align*}
\end{enumerate}
\end{homeworkProblem}
\newpage
\begin{homeworkProblem}[4]
\solution
\begin{enumerate}[(a)]
    \item 
    Assume $X,Y\sim N(0,1)$, and $M_1=max(X,Y)$, $M_2=min(X,Y)$.\\
    From the def we can have $X+Y=M_1+M_2$ and $XY=M_1M_2$ and $M_1-M_2=\left\lvert X-Y \right\rvert $.\\
    And what we want to get is $Corr(M_1,M_2)=\frac{Cov(M_1,M_2)}{\sqrt{var(M_1)var(M_2)}}$.\\
    So first we calculate $Cov(M_1,M_2)=E[M_1M_2]-E[M_1]E[M_2]$.\\
    And from what we have, we can get $E[M_1M_2]=E[XY]=E[X]E[Y]=0*0=0$.\\
    So we just need to calculate $E[M_1],E[M_2]$.\\
    \begin{align}
        E[M_1+M_2]=E[M_1]+E[M_2]=E[X+Y]=E[X]+E[Y]=0+0=0\\
        E[M_1-M_2]=E[M_1]-E[M_2]=E[\left\lvert X-Y \right\rvert]
    \end{align}
    From the property of symmetry, we have $-Y\sim N(0,1)$.\\
    And then $X-Y\sim N(0,2)$, which can be proved by MGF.\\
    And then we can denote it as $X-Y=\sqrt{2}Z$, in which $Z\sim N(0,1)$.\\
    Then we can have $E[\left\lvert X-Y \right\rvert]=E[\left\lvert \sqrt{2}Z \right\rvert]=\sqrt{2}E[\left\lvert Z \right\rvert]=2\sqrt{2}\int_{0}^{\infty} \frac{1}{\sqrt{2\pi}}e^{-\frac{z^2}{2}} \,dz=\frac{2}{\sqrt{\pi}} $.\\
    Therefore we can get $E[M_1]=\frac{1}{\sqrt{\pi}}$, $E[M_2]=-\frac{1}{\sqrt{\pi}}$.\\
    Thereby we can get $Cov(M_1,M_2)=\frac{1}{\pi}$.\\
    Next what we need to calculate is $var(M_1)=E[M_1^2]-E[M_1]^2=E[M_1^2]-\frac{1}{\pi}$, $var(M_2)=E[M_2^2]-\frac{1}{\pi}$.\\
    Because of the symmetry property, we can have $var(M_1)=var(M_2)$ and let us assume it as $x$.\\
    \begin{align}
        E[(X-Y)^2]-E[X-Y]^2=E[(X-Y)^2]-0=var(X-Y)=2\\
        E[M_1^2]+E[M_2^2]-2E[M_1M_2]=E[\left\lvert X-Y \right\rvert^2]-0=E[(X-Y)^2]
    \end{align}
    Then we can get $x=1-\frac{1}{\pi}$.\\
    So the final answer is $Corr=\frac{\frac{1}{\pi}}{1-\frac{1}{\pi}}=\frac{1}{\pi -1}$. 
\end{enumerate}
\end{homeworkProblem}
\newpage
\begin{homeworkProblem}[5]
\solution
\begin{enumerate}[(a)]
    \item 
    From the def of Expectation, then we have $E[X]=\overline{x}$, $E[Y]=\overline{y}$.\\
    And $P(X=x_i,Y=y_i)=\frac{1}{n}$ because we choose one pair from n pairs uniformly in random.\\
    Then, 
    \begin{align*}
        Cov(X,Y)&=E[(X-EX)(Y-EY)]\\
        &=E[(X-\overline{x})(Y-\overline{y})]\\
        &=\sum_{i=1}^{n}P(X=x_i,Y=y_i)(x_i-\overline{x})(y_i-\overline{y})\\
        &=\frac{1}{n}\sum_{i=1}^{n}(x_i-\overline{x})(y_i-\overline{y})\\
        &=r
    \end{align*}
    Therefore, $Cov(X,Y)$ is related to the sample variance.
\end{enumerate}
\begin{enumerate}[(b)]
    \item
    First the total signed area is $S=\sum_{i<j}(x_i-x_j)(y_i-y_j)$.\\
    Also $E[(X-\widetilde{X})(Y-\widetilde{Y})]=E[XY]+E[\widetilde{X}\widetilde{Y}]-E[X\widetilde{Y}]-E[Y\widetilde{X}]$.\\
    Because $XY$ and $\widetilde{X}\widetilde{Y}$ are the same distribution, and $X$ and $\widetilde{Y}$ are independent and similarly for $\widetilde{X}$ and $Y$.\\
    Then we can have $E[(X-\widetilde{X})(Y-\widetilde{Y})]=2E[XY]-2E[X]E[Y]=2Cov(X,Y)=2r$.\\
    Then we can consider $E[(X-\widetilde{X})(Y-\widetilde{Y})]=\frac{1}{n^2}\sum_{i=1}^{n}\sum_{j=1}^{n}(x_i-x_j)(y_i-y_j)$.\\
    And $E[(X-\widetilde{X})(Y-\widetilde{Y})]=\frac{n*0+2\sum_{i<j}(x_i-x_j)(y_i-y_j)}{n^2}=\frac{2S}{n^2}$.\\
    Then $S=n^2r$.
\end{enumerate}
\begin{enumerate}[(c)]
    \item
    (i)Reeversing the axes does not affect the area of the rectangle.\\
    (ii)Scaling the width and weight along its axis just changes the area by the same factor. Then we just need to multiply a1 and a2.\\
    (iii)Shifting will not influence the area of rectangle, because the length is relative.\\ 
    (iv)A rectangle, whose width is a and height is b+c, can be divided into two small rectangles. One is $a*b$, and another is $b*c$. Obviously the area of the big original rectangle is equal to that of the two small rectangles. At the same time, A positive-area rectangle is divided into two positive-area rectangles and a negative-area rectangle is divided into two negative-area rectangles.
\end{enumerate}
\end{homeworkProblem}    
\end{document}
