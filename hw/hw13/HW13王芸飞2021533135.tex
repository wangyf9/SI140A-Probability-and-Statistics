\documentclass{article}
\usepackage{graphicx}
\usepackage{fancyhdr}
\usepackage{extramarks}
\usepackage{amsmath}
\usepackage{amsthm}
\usepackage{amsfonts}
\usepackage{tikz}
\usepackage[plain]{algorithm}
\usepackage{algpseudocode}
\usepackage{enumerate}
\usepackage{tikz}
\usepackage{xifthen}
\usepackage{xparse}
\usepackage{listings}
\usepackage{amsmath, amssymb}
\usepackage{subfigure}
\usepackage{lipsum}

\usetikzlibrary{automata,positioning}

%
% Basic Document Settings
%  

\topmargin=-0.45in
\evensidemargin=0in
\oddsidemargin=0in
\textwidth=6.5in
\textheight=9.0in
\headsep=0.25in

\linespread{1.1}

\pagestyle{fancy}
\lhead{\hmwkAuthorName}
\chead{\hmwkClass : \hmwkTitle}
\rhead{\firstxmark}
\lfoot{\lastxmark}
\cfoot{\thepage}

\renewcommand\headrulewidth{0.4pt}
\renewcommand\footrulewidth{0.4pt}

\setlength\parindent{0pt}

%
% Create Problem Sections
%

\newcommand{\enterProblemHeader}[1]{
    \nobreak\extramarks{}{Problem \arabic{#1} continued on next page\ldots}\nobreak{}
    \nobreak\extramarks{Problem \arabic{#1} (continued)}{Problem \arabic{#1} continued on next page\ldots}\nobreak{}
}

\newcommand{\exitProblemHeader}[1]{
    \nobreak\extramarks{Problem \arabic{#1} (continued)}{Problem \arabic{#1} continued on next page\ldots}\nobreak{}
    \stepcounter{#1}
    \nobreak\extramarks{Problem \arabic{#1}}{}\nobreak{}
}

\newcommand*\circled[1]{\tikz[baseline=(char.base)]{
		\node[shape=circle,draw,inner sep=2pt] (char) {#1};}}


\setcounter{secnumdepth}{0}
\newcounter{partCounter}
\newcounter{homeworkProblemCounter}
\setcounter{homeworkProblemCounter}{1}
\nobreak\extramarks{Problem \arabic{homeworkProblemCounter}}{}\nobreak{}

%
% Homework Problem Environment
%
% This environment takes an optional argument. When given, it will adjust the
% problem counter. This is useful for when the problems given for your
% assignment aren't sequential. See the last 3 problems of this template for an
% example.
%

\newenvironment{homeworkProblem}[1][-1]{
    \ifnum#1>0
        \setcounter{homeworkProblemCounter}{#1}
    \fi
    \section{Problem \arabic{homeworkProblemCounter}}
    \setcounter{partCounter}{1}
    \enterProblemHeader{homeworkProblemCounter}
}{
    \exitProblemHeader{homeworkProblemCounter}
}

%
% Homework Details
%   - Title
%   - Class
%   - Due date
%   - Name
%   - Student ID

\newcommand{\hmwkTitle}{Homework\ \#13}
\newcommand{\hmwkClass}{Probability \& Statistics for EECS}
\newcommand{\hmwkDueDate}{May 14, 2023}
\newcommand{\hmwkAuthorName}{Wang Yunfei}
\newcommand{\hmwkAuthorID}{2021533135}


%
% Title Page
%

\title{
    \vspace{2in}
    \textmd{\textbf{\hmwkClass:\\  \hmwkTitle}}\\
    \normalsize\vspace{0.1in}\small{Due\ on\ \hmwkDueDate\ at 23:59}\\
	\vspace{4in}
}

\author{
	Name: \textbf{\hmwkAuthorName} \\
	Student ID: \hmwkAuthorID}
\date{}

\renewcommand{\part}[1]{\textbf{\large Part \Alph{partCounter}}\stepcounter{partCounter}\\}

%
% Various Helper Commands
%

% Useful for algorithms
\newcommand{\alg}[1]{\textsc{\bfseries \footnotesize #1}}
% For derivatives
\newcommand{\deriv}[1]{\frac{\mathrm{d}}{\mathrm{d}x} (#1)}
% For partial derivatives
\newcommand{\pderiv}[2]{\frac{\partial}{\partial #1} (#2)}
% Integral dx
\newcommand{\dx}{\mathrm{d}x}
% Alias for the Solution section header
\newcommand{\solution}{\textbf{\large Solution}}
% Probability commands: Expectation, Variance, Covariance, Bias
\newcommand{\E}{\mathrm{E}}
\newcommand{\Var}{\mathrm{Var}}
\newcommand{\Cov}{\mathrm{Cov}}
\newcommand{\Bias}{\mathrm{Bias}}

\begin{document}

\maketitle

\pagebreak

\begin{homeworkProblem}[1]
\solution
\begin{enumerate}[(a)]
    \item 
   First for any $X_j$, we have, 
   \begin{align*}
        P(X_j>=1)&=P(X_j>1)\\
        &=1-P(X_j<=1)\\
        &=1-(1-\frac{1}{e})\\
        &=\frac{1}{e}
   \end{align*}
   And then from the question, N is the r.v. which represents the fisrt $X_j$ exceeds 1.\\
   Therefore we can seem N as the Fisrt success distribution, and the parameter $p=\frac{1}{e}$.\\
   Then from what we have learned, we know N-1 actually is distributed as Geometric distribution, \\
   that is $N-1\sim Geom(\frac{1}{e})$.\\
   At the meanwhile, $E[N]=p=e$.\\
\end{enumerate}
\begin{enumerate}[(b)]
    \item 
    First in this question, we should think of the relationship between Poisson distribution and Exponential distribution. From what we have learned, the r.v. M-1 can be seemed as during the time ranging from 0 to 10, how many arrivals occurs in this period.\\
    And because $X_1,X_2,\cdots \sim Expo(1)$ and $t=10$. Then $\lambda t=10$\\
    Then this question actually is the number of arrivals that occurs in an interval of length 10, that is $M-1\sim Pois(10)$, because $M$ actually has exceeded or been equal to 10.\\
    Then $E[M-1]=E[M]-1=10$.\\
    Therefore $E[M]=11$.

\end{enumerate}
\begin{enumerate}[(c)]
    \item
    First, for $\frac{X_j}{n}$, let $Y=\frac{X_j}{n}$. Then $P(Y<=x)=P(X_j<=nx)=1-e^{-nx}$. Therefore we can get $Y=\frac{X_j}{n}\sim Expo(n)$.\\
    Second, from the theorem we learned about Gamma distribution, if $X_1,\cdots X_n\sim Expo(\lambda)$ and they are i.i.d.s, then we can have $X_1+\cdots+X_n\sim Gamma(n,\lambda)$.\\
    Then we can have $\overline{X_n}\sim Gamma(n,n) $.\\
    And from the properties of Gamma distribution, we have $E[\overline{X_n}]=1, Var[\overline{X_n}]=\frac{1}{n}$.
    And from the properties of Exponential distribution, we have $E[X_j]=\frac{1}{\lambda}=1=\mu , Var[X_j]=\frac{1}{\lambda^2}=1=\sigma ^2$.\\
    Then we have $E[\overline{X_n}]=\mu, Var[\overline{X_n}]=\frac{\sigma^2}{n}$.\\
    Then we can use CLT to get the right answer, that is when $n\rightarrow \infty,\ \overline{X_n}\sim N(1,\frac{1}{n})$.
\end{enumerate}
\end{homeworkProblem}
\newpage
\begin{homeworkProblem}[2]

\begin{enumerate}[(a)]
    \item
    Denote $\overline{X}=\frac{1}{n}\sum_{i=1}^{n}X_i$.\\
    Let $S_n=\sum_{i=1}^{n}X_i$ and when $s>0,t>0$, we can get the following formula from chernoff's ineuqality,
    \begin{align*}
        P(S_n-E[S_n]\geq t)%%&=P(e^{s(S_n-E[S_n])}\geq e^{st})\\
        &\leq e^{-st}E[e^{s(S_n-E[S_n])}]\\
        &=e^{-st}\prod_{i=1}^nE[e^{s(X_i-E[X_i])}]\\
    from\ the\ Hoeffding\ Lemma\ ,\ we\ can\ have\\
    P(S_n-E[S_n]\geq t)&\leq  e^{-st}\prod_{i=1}^ne^{\frac{s^2(b_i-a_i)^2}{8}}\\
    &=e^{-st+\frac{1}{8}s^2\sum_{i=1}^{n}(b_i-a_i)^2}
    \end{align*} 
     Because we need to find the min value of $ e^{-st+\frac{1}{8}s^2\sum_{i=1}^{n}(b_i-a_i)^2}$, then let $f(s)=-st+\frac{1}{8}s^2\sum_{i=1}^{n}(b_i-a_i)^2$, and we need to find the the min of $f(s)$.\\
     Obviously $f^2(x)>=0$, then we just need to find $f'(s)=0$.\\
     Finally we can get $s=\frac{4t}{\sum_{i=1}^{n}(b_i-a_i)^2}$, and then substitute into equation, we can get\\
     \begin{align*}
        P(S_n-E[S_n]\geq t)\leq e^{\frac{-2t^2}{\sum_{i=1}^{n}(b_i-a_i)^2}}
     \end{align*}
     And because $\overline{X}=\frac{S_n}{n}$, then we can get,
     \begin{align*}
        P(\overline{X}-E[\overline{X}]\geq t)\leq e^{\frac{-2n^2t^2}{\sum_{i=1}^{n}(b_i-a_i)^2}}
     \end{align*}
     Similarly we can get $ P(E[\overline{X}]-\overline{X}\geq t)\leq e^{\frac{-2n^2t^2}{\sum_{i=1}^{n}(b_i-a_i)^2}}$.\\
     Then we can get $ P(\left\lvert \overline{X}-E[\overline{X}]\right\rvert \geq t)\leq e^{\frac{-2n^2t^2}{\sum_{i=1}^{n}(b_i-a_i)^2}}$.\\
     And because $X_1,X_2,\cdots,X_n$ are independent with $E[X_I]=\mu$, $a\leq X_i\leq b$ for $i=1,\cdots,n$.\\
     Therefore we can get the final answer, 
     \begin{align*}
        P(\overline{X}-E[\overline{X}]\geq t)&\leq e^{\frac{-2n^2t^2}{n(b_i-a_i)^2}}\\
            &\leq e^{\frac{-2nt^2}{(b_i-a_i)^2}}
     \end{align*}
\end{enumerate}
\end{homeworkProblem}
\newpage
\begin{homeworkProblem}[3]
\solution
\begin{enumerate}[(a)]
    \item
    First for any $k\geq 0, a \geq 0$, we can have
    \begin{align*}
        P(X-\mu \geq a)&=P(X-\mu+k \geq a+k)\\
        From\ the\ def\ of\ Probability\ then\ we\ can\ have\ 
        &\leq P((X-\mu+k )^2 \geq (a+k)^2 )\\
    \end{align*}
    Then we can use Markov's ineuqality to get the following expression for any $k\geq 0, a \geq 0$,
    \begin{align*}
        P((X-\mu+k )^2 \geq (a+k)^2 )&=P(\left\lvert (X-\mu+k )^2\right\rvert \geq (a+k)^2)\\
        &\leq \frac{E[\left\lvert (X-\mu+k )^2\right\rvert]}{(a+k)^2}\\
        &\leq \frac{E[X^2+\mu^2+k^2-2X\mu +2Xk-2\mu k]}{(a+k)^2}\\
        &\leq\frac{E[X^2]+E[\mu^2]+E[k^2]-2E[X\mu]-2E[\mu k]+2E[Xk]}{(a+k)^2}\\
        &\leq\frac{E[X^2]+\mu^2+k^2-2\mu\mu-2\mu k+2\mu k}{(a+k)^2}\\
        &\leq\frac{E[X^2]-E[X]^2+k^2}{(a+k)^2}\\
        &\leq\frac{\sigma^2+k^2}{(a+k)^2}\\
    \end{align*}
    So let us denote $\frac{\sigma^2+k^2}{(a+k)^2}=g(k)$, and then what we need to do is to find the minimal value of it.\\
    Therefore $g'(k)=\frac{2(ak-\sigma^2)}{(a+k)^3}$, and because $g^2(k)>=0$, then let $g'(k)=0$, we can get the minimum.\\
    Then $k=\frac{\sigma^2}{a}$, at the meanwhile $g(k)=\frac{a^2+k^2}{a^2+k^2+2\sigma^2}$.\\
    Because $k>=0$, and from what we have know about $\frac{a}{b}<=\frac{a+1}{b+1}$ for $a,b>=1$.\\
    Then let $k=0$, we can get $g(0)=\frac{a^2}{a^2+2\sigma^2}$.\\
    Therefore $P(X-\mu \geq a)<=\frac{a^2}{a^2+2\sigma^2}<=\frac{a^2}{a^2+\sigma^2}$.\\
    Finally we get it.
\end{enumerate}
\end{homeworkProblem}
\newpage
\begin{homeworkProblem}[4]
\solution
\begin{enumerate}[(a)]
    \item 
    First from the question, we have prior distribution denoted as 
    \begin{align*}
        f_{\Theta}(\theta)=\frac{1}{\sqrt{2\pi}\sigma_0}e^{\frac{-(\theta-x_0)^2}{2\sigma_0^2}}
    \end{align*}
    and we also have the collection of data denoted as 
    \begin{align*}
        f_{X|\Theta}(x)=\prod_{i=1}^{n}\frac{1}{\sqrt{2\pi}\sigma_i}e^{\frac{-(x_i-\theta)^2}{2\sigma_i^2}}
    \end{align*}
    Then $x_1,x_2,\cdots,x_n\in R$(Observation), $x_i\in R$ and $x=(x_1,\cdots,x_n)$.\\
    %%Thereby, we can get,
    %%\begin{align*}
      %%  f(X|\Theta)=\prod_{i=1}^{n}\frac{1}{\sqrt{2\pi}\sigma_i}e^{\frac{-(x_i-\Theta)^2}{2\sigma_i^2}}
    %%\end{align*}
    Next, we can use Bayes' rule to get the posterior PDF of $\Theta$ denoted as $f_{\Theta|X}(\theta)$.
    \begin{align*}
        f_{\Theta|X}(\theta)=\frac{f_{X|\Theta}(x)f_{\Theta}(\theta)}{f_{X}(x)}
    \end{align*}
    But $f_{X}(x)$ actually is not related with parameter $\theta$, because we need to use LOTP rule to integrate it to get $f_{X}(x)$.\\
    Then we can denote it as a constant $c$, and get the following expression,
    \begin{align*}
        f_{\Theta|X}(\theta)&=cf_{X|\Theta}(x)f_{\Theta}(\theta)\\
        &=c*\frac{1}{\sqrt{2\pi}\sigma_0}e^{\frac{-(\theta-x_0)^2}{2\sigma_0^2}}*\prod_{i=1}^{n}\frac{1}{\sqrt{2\pi}\sigma_i}e^{\frac{-(x_i-\theta)^2}{2\sigma_i^2}}\\
        &=c*\frac{1}{(\sqrt{2\pi})^{n+1}}*\frac{1}{\prod_{i=0}^{n}\sigma_i}*e^{-\sum_{j=0}^{n}\frac{(\theta-x_j)^2}{2\sigma_j^2}}
    \end{align*}
    Then we can simplify it by denoting $c*\frac{1}{(\sqrt{2\pi})^{n+1}}*\frac{1}{\prod_{i=0}^{n}\sigma_i}$ as $C$.\\
    Then we take the log of both sides and get the following expression,
    \begin{align*}
        logf_{\Theta|X}(\theta)=C-\sum_{i=0}^{n}\frac{(\theta-x_i)^2}{2\sigma_i^2}
    \end{align*}
    Next step we need to derive $\theta$ by derivation.
    Denote $g(\theta)=C-\sum_{i=0}^{n}\frac{(\theta-x_i)^2}{2\sigma_i^2}$, then we can get $\theta=\frac{\sum_{i=0}^{n}\frac{x_i}{\sigma_i^2}}{\sum_{j=0}^{n}\frac{1}{\sigma_j^2}}$ by letting $g'(\theta)=0$.\\
   %% And then we can similarly get new variance $\sigma^2=\frac{1}{{\sum_{i=0}^{n}\frac{1}{\sigma_i^2}}}$.\\
    Finally we can get the posterior PDF of $\Theta$,
    \begin{align*}
        f_{\Theta|X}(\theta)=C*e^{-\sum_{i=0}^{n}\frac{(\theta-x_i)^2}{2\sigma_i^2}}
    \end{align*}
    and the $\theta,C$ is what we have calculated.
    And then we can simplify the expression and get it is a normal distribution with the mean $\mu$ and the variance $\sigma^2$.\\
    $\mu=\frac{\sum_{i=0}^{n}\frac{x_i}{\sigma_i^2}}{\sum_{j=0}^{n}\frac{1}{\sigma_j^2}}$ and $\frac{1}{\sigma^2}=\frac{1}{\prod_{i=0}^{n}\sigma_i^2}$.
\end{enumerate}
\end{homeworkProblem}
\newpage
\begin{homeworkProblem}[5]
\solution
\begin{enumerate}[(a)]
    \item 
    First n independent Exponential distribution $X_1, X_2, \cdots X_n \sim Expo(\theta)$, $f_{X_i}(x)=\theta e^{-\theta x}$, $x\in (0,+\infty)$.\\
    Then $x_1,x_2,\cdots,x_n\in R$(Observation), $x_i\in (0,+\infty)$.\\
    And then,
    \begin{align*}
        f_X(x;\theta)&=\prod_{i=1}^{n}f_{X_i}(x_i;\theta)\\
        &=\prod_{i=1}^{n}\theta e^{-\theta x_i}\\
        &=\theta^ne^{-\theta (x_1+x_2+\cdots+x_n)}
    \end{align*}
    Denote $x_1+x_2+\cdots+x_n=S_n$, and then we can have
    \begin{align*}
        log f_X(x;\theta)=log \theta^ne^{-\theta S_n}=f(\theta)
    \end{align*}
    Then we can get $f'(\theta)=\frac{n}{\theta}-S_n$, and $f^2(\theta)=-\frac{n}{\theta^2}\leq 0$.\\
    Therefore $f'(\theta)=0$, then we can get the max value of $f(\theta)$ and at the meanwhile $\theta=\frac{n}{S_n}$.\\
    Then $\overline{\theta_{MLE}}=argmax_{\theta}f(\theta)=\frac{n}{S_n}=\frac{n}{x_1+x_2+\cdots+x_n}$.
\end{enumerate}
\begin{enumerate}[(b)]
    \item
    First n independent normal distribution $X_1, X_2, \cdots X_n \sim N(\mu,v)$, $f_{X_i}(x)=\frac{1}{\sqrt{2\pi v}} e^{-\frac{(x-\mu)^2}{2v}}$, $x\in R$.\\
    Then $x_1,x_2,\cdots,x_n\in R$(Observation), $x_i\in R$.\\
    And then,
    \begin{align*}
        f_X(x;\mu,v)&=\prod_{i=1}^{n}f_{X_i}(x_i;\mu,v)\\
        &=\prod_{i=1}^{n}\frac{1}{\sqrt{2\pi v}} e^{-\frac{(x-\mu)^2}{2v}}\\
        &=(2\pi v)^{-\frac{n}{2}}e^{-\frac{\sum_{i=1}^{n}(x_i-\mu)^2}{2v}}
    \end{align*}
    Then we can have,
    \begin{align*}
        log f_X(x;\theta)&=log ((2\pi v)^{-\frac{n}{2}}e^{-\frac{\sum_{i=1}^{n}(x_i-\mu)^2}{2v}})\\
        &=-\frac{n}{2}log(2\pi) -\frac{n}{2}log(v)-\frac{\sum_{i=1}^{n}(x_i-\mu)^2}{2v}
    \end{align*}
    Then we need divide it into two steps, one is to take the partial derivative with respect to $\mu$, and another is $v$.\\
    Then denote $g(v)=-\frac{n}{2}log(2\pi) -\frac{n}{2}log(v)-\frac{\sum_{i=1}^{n}(x_i-\mu)^2}{2v}=h(\mu)$.\\
    And then let $g'(v)=0=\frac{\sum_{i=1}^{n}(x_i-\mu)^2}{2v^2}-\frac{n}{2v}$ and $h'(\mu)=0=\frac{\sum_{i=1}^{n}(x_i-\mu)}{v}$, \\
    finally we can get $v=\frac{\sum_{i=1}^{n}(x_i-\mu)^2}{n}$, $\mu=\frac{\sum_{i=1}^{n}x_i}{n}$.\\
    In the end $\overline{v_{MLE}}=argmax_{v}g(v)=\frac{\sum_{i=1}^{n}(x_i-\mu)^2}{n}$,\\
    and $\overline{\mu_{MLE}}=argmax_{\mu}h(\mu)=\frac{\sum_{i=1}^{n}x_i}{n}$.\\
    Therefore $\overline{\theta}=(\frac{\sum_{i=1}^{n}x_i}{n},\frac{\sum_{i=1}^{n}(x_i-\mu)^2}{n})$.
\end{enumerate}

\end{homeworkProblem}    
\end{document}
