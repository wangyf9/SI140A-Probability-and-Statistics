\documentclass{article}

\usepackage{fancyhdr}
\usepackage{extramarks}
\usepackage{amsmath}
\usepackage{amsthm}
\usepackage{amsfonts}
\usepackage{tikz}
\usepackage[plain]{algorithm}
\usepackage{algpseudocode}
\usepackage{enumerate}
\usepackage{tikz}
\usepackage{xifthen}
\usepackage{xparse}
\usepackage{amsmath, amssymb}
\usepackage{lipsum}
\usetikzlibrary{automata,positioning}

%
% Basic Document Settings
%  

\topmargin=-0.45in
\evensidemargin=0in
\oddsidemargin=0in
\textwidth=6.5in
\textheight=9.0in
\headsep=0.25in

\linespread{1.1}

\pagestyle{fancy}
\lhead{\hmwkAuthorName}
\chead{\hmwkClass : \hmwkTitle}
\rhead{\firstxmark}
\lfoot{\lastxmark}
\cfoot{\thepage}

\renewcommand\headrulewidth{0.4pt}
\renewcommand\footrulewidth{0.4pt}

\setlength\parindent{0pt}

%
% Create Problem Sections
%

\newcommand{\enterProblemHeader}[1]{
    \nobreak\extramarks{}{Problem \arabic{#1} continued on next page\ldots}\nobreak{}
    \nobreak\extramarks{Problem \arabic{#1} (continued)}{Problem \arabic{#1} continued on next page\ldots}\nobreak{}
}

\newcommand{\exitProblemHeader}[1]{
    \nobreak\extramarks{Problem \arabic{#1} (continued)}{Problem \arabic{#1} continued on next page\ldots}\nobreak{}
    \stepcounter{#1}
    \nobreak\extramarks{Problem \arabic{#1}}{}\nobreak{}
}

\newcommand*\circled[1]{\tikz[baseline=(char.base)]{
		\node[shape=circle,draw,inner sep=2pt] (char) {#1};}}


\setcounter{secnumdepth}{0}
\newcounter{partCounter}
\newcounter{homeworkProblemCounter}
\setcounter{homeworkProblemCounter}{1}
\nobreak\extramarks{Problem \arabic{homeworkProblemCounter}}{}\nobreak{}

%
% Homework Problem Environment
%
% This environment takes an optional argument. When given, it will adjust the
% problem counter. This is useful for when the problems given for your
% assignment aren't sequential. See the last 3 problems of this template for an
% example.
%

\newenvironment{homeworkProblem}[1][-1]{
    \ifnum#1>0
        \setcounter{homeworkProblemCounter}{#1}
    \fi
    \section{Problem \arabic{homeworkProblemCounter}}
    \setcounter{partCounter}{1}
    \enterProblemHeader{homeworkProblemCounter}
}{
    \exitProblemHeader{homeworkProblemCounter}
}

%
% Homework Details
%   - Title
%   - Class
%   - Due date
%   - Name
%   - Student ID

\newcommand{\hmwkTitle}{Homework\ \#09}
\newcommand{\hmwkClass}{Probability \& Statistics for EECS}
\newcommand{\hmwkDueDate}{April 16, 2023}
\newcommand{\hmwkAuthorName}{Wang Yunfei}
\newcommand{\hmwkAuthorID}{2021533135}


%
% Title Page
%

\title{
    \vspace{2in}
    \textmd{\textbf{\hmwkClass:\\  \hmwkTitle}}\\
    \normalsize\vspace{0.1in}\small{Due\ on\ \hmwkDueDate\ at 23:59}\\
	\vspace{4in}
}

\author{
	Name: \textbf{\hmwkAuthorName} \\
	Student ID: \hmwkAuthorID}
\date{}

\renewcommand{\part}[1]{\textbf{\large Part \Alph{partCounter}}\stepcounter{partCounter}\\}

%
% Various Helper Commands
%

% Useful for algorithms
\newcommand{\alg}[1]{\textsc{\bfseries \footnotesize #1}}
% For derivatives
\newcommand{\deriv}[1]{\frac{\mathrm{d}}{\mathrm{d}x} (#1)}
% For partial derivatives
\newcommand{\pderiv}[2]{\frac{\partial}{\partial #1} (#2)}
% Integral dx
\newcommand{\dx}{\mathrm{d}x}
% Alias for the Solution section header
\newcommand{\solution}{\textbf{\large Solution}}
% Probability commands: Expectation, Variance, Covariance, Bias
\newcommand{\E}{\mathrm{E}}
\newcommand{\Var}{\mathrm{Var}}
\newcommand{\Cov}{\mathrm{Cov}}
\newcommand{\Bias}{\mathrm{Bias}}

\begin{document}

\maketitle

\pagebreak

\begin{homeworkProblem}[1]
\solution
\begin{enumerate}[(a)]
    \item
    X,Y are discrete
    \begin{align*}
    P(Y=y|X=x)&=\frac{P(Y=y,X=x)}{P(X=x)}\\
              &=\frac{P(X=x|Y=y)P(Y=y)}{P(X=x)}    
    \end{align*}
    This uses Bayes rule and the confine of it is $P(Y=y),P(X=x)>0$.
\end{enumerate}
\begin{enumerate}[(b)]
    \item
    X is discrete, Y is continuous
    \begin{align*}
    P(Y\in (y-\varepsilon ,y+\varepsilon )|X=x)&=\frac{P(X=x,Y\in (y-\varepsilon ,y+\varepsilon ))}{P(X=x)}\\
              &=\frac{P(X=x|Y\in (y-\varepsilon ,y+\varepsilon ))P(Y\in (y-\varepsilon ,y+\varepsilon ))}{P(X=x)}\\
              \frac{P(Y\in (y-\varepsilon ,y+\varepsilon )|X=x)}{2\varepsilon }&=\frac{P(X=x|Y\in (y-\varepsilon ,y+\varepsilon ))\frac{P(Y\in (y-\varepsilon ,y+\varepsilon ))}{2\varepsilon }}{P(X=x)}\\
    Because\ the\ definition\ of\ integral\ and\ when\ \varepsilon =0 \\
            f_Y(y|X=x)&=\frac{P(X=x|Y=y)f_Y(y)}{P(X=x)}
    \end{align*}
    This uses Bayes rule and the confine of it is $P(Y\in (y-\varepsilon ,y+\varepsilon )),P(X=x)>0$.
   
\end{enumerate}
\begin{enumerate}[(c)]
    \item
    X is continuous, Y is discrete
    \begin{align*}
    P(Y=y|X\in (x-\varepsilon ,x+\varepsilon ))&=\frac{P(X\in (x-\varepsilon ,x+\varepsilon ),Y=y)}{P(X\in (x-\varepsilon ,x+\varepsilon ))}\\
              &=\frac{P(X\in (x-\varepsilon ,x+\varepsilon )|Y=y)P(Y=y)}{P(X\in (x-\varepsilon ,x+\varepsilon ))}\\
              P(Y\in (y-\varepsilon ,y+\varepsilon )|X=x)&=\frac{\frac{P(X\in(x-\varepsilon ,x+\varepsilon )|Y=y)}{2\varepsilon }P(Y=y) }{\frac{P(X\in (x-\varepsilon ,x+\varepsilon ))}{2\varepsilon }}\\
    Because\ the\ definition\ of\ integral\ and\ when\ \varepsilon =0 \\
    P(Y=y|X=x)&=\frac{f_X(x|Y=y)P(Y=y)}{f_X(x)}
    \end{align*}
    This uses Bayes rule and the confine of it is $P(Y=y),P(X\in (x-\varepsilon ,x+\varepsilon ))>0$.
\end{enumerate}
\begin{enumerate}[(d)]
    \item
    X,Y are continuous
    \begin{align*}
    P(Y\in (y-\varepsilon ,y+\varepsilon )|X\in (x-\varepsilon ,x+\varepsilon ))&=\frac{P(Y\in (y-\varepsilon ,y+\varepsilon ),X\in (x-\varepsilon ,x+\varepsilon ))}{P(X\in (x-\varepsilon ,x+\varepsilon ))}\\
    \frac{P(Y\in (y-\varepsilon ,y+\varepsilon )|X\in (x-\varepsilon ,x+\varepsilon ))}{2\varepsilon }&=\frac{\frac{P(Y\in (y-\varepsilon ,y+\varepsilon ),X\in (x-\varepsilon ,x+\varepsilon ))}{4\varepsilon^2}}{\frac{P(X\in (x-\varepsilon ,x+\varepsilon ))}{2\varepsilon }}\\
    Because\ the\ definition\ of\ integral\ and\ when\ \varepsilon =0 \\
    f_{Y|X}(y|x)&=\frac{f_{X,Y}(x,y)}{f_X(x)}\\
                &=\frac{f_{X|Y(x|y)f_Y(y)}}{f_X(x)}
    \end{align*}
    This uses Bayes rule and the confine of it is $P(Y\in (y-\varepsilon ,y+\varepsilon )),P(X\in (x-\varepsilon ,x+\varepsilon ))>0$.
\end{enumerate}
\end{homeworkProblem}
\newpage
\begin{homeworkProblem}[2]

\begin{enumerate}[(a)]
    \item
    Because $N=X+Y$, then $P(N=x+y|X=x,Y=y)=1$ and we can have:\\
    $P(X=x,Y=y,N=n)=P(X=x,Y=y)$\\
    Because $X$ and $Y$ are i.i.d., then we can have:\\
    $P(X=x,Y=y)=P(X=x)P(Y=y)=p*q^x*p*q^y=p^2*q^{x+y}=p^2q^n$, for nonnegative integers $x,y,n$, and $n=x+y$.\\
\end{enumerate}
\begin{enumerate}[(b)]
    \item
    From(1), we can also have:
    \begin{align*}
        P(X=x,N=n)&=P(X=x,Y=n-x,N=n)\\
        &=P(X=x,Y=n-x)\\
        &=P(X=x)P(Y=n-x)\\
        &=p*q^x*p*q^{n-x}=p^2q^n
    \end{align*}
    For $0<=x<=n$.
\end{enumerate}
\begin{enumerate}[(c)]
    \item
    \begin{align*}
        Bayes\ rule\\
        P(X=x|N=n)&=\frac{P(X=x,N=n)}{P(N=n)}\\
        LOTP\\
                  &=\frac{p^2q^n}{\sum_{x=0}^{n}P(N=n|X=x)P(X=x)}\\
        N=X+Y\\
                  &=\frac{p^2q^n}{\sum_{x=0}^{n}P(X+Y=n|X=x)P(X=x)}\\
                  &=\frac{p^2q^n}{\sum_{x=0}^{n}P(Y=n-x|X=x)P(X=x)}\\
        Because\ of\ X,Y\ are\ i.i.d.\\
                  &=\frac{p^2q^n}{\sum_{x=0}^{n}P(Y=n-x)P(X=x)}\\
                  &=\frac{p^2q^n}{\sum_{x=0}^{n}p^2q^n}\\
                  &=\frac{p^2q^n}{(n+1)p^2q^n}\\
                  &=\frac{1}{n+1}
    \end{align*}
    At the same time, obviously we can get $N\sim NBin(2,p)$. That is because $X,Y$ are i.i.d. $Geom(p)$, and N=X+Y, then we 
    can easily find that all we need to calculate is the number of the failture times before the second success. So we need to find one success and n failures in the first n+1 times. Then we have $P(N=n)=\binom{n+2-1}{2-1}p*p*q^n=(n+1)p^2q^n$. Therefore we can also get it.\\
    Moreover, when N is given, $P(X=x|N=n)=\frac{1}{n+1}$ is exact.
\end{enumerate}
\end{homeworkProblem}
\newpage
\begin{homeworkProblem}[3]
\solution
\begin{enumerate}[(a)]
    \item
    From the question, we have
    \begin{align*}
        P(X<=x|X>c)&=\frac{P(X<=x,X>c)}{P(X>c)}\\
                   &=\frac{P(c<X<=x)}{P(X>c)}\\
                   &=\frac{P(X<=x)-P(X<=c)}{1-P(X<=c)}\\
                   &=\frac{1-e^{-\lambda x}-1+e^{-\lambda c}}{1-1+e^{-\lambda c}}\\
                   &=1-e^{-\lambda (x-c)}
    \end{align*}
    Then the conditional PDF of X given $X > c$ is $\lambda e^{-\lambda (x-c)}$. 
\end{enumerate}
\begin{enumerate}[(b)]
    \item
    First if $x<=c$, then $P(X<=x|X<c)=1$.\\
    On the other hand when $x>c$, we have
    \begin{align*}
        P(X<=x|X<c)&=\frac{P(X<=x,X<c)}{P(X<c)}\\
                   &=\frac{P(X<=x)}{P(X<c)}\\
                   &=\frac{1-e^{-\lambda x}}{1-e^{-\lambda c}}\\
    \end{align*}
    Then the conditional PDF of X given $X < c$ is $\frac{\lambda}{1-e^{-\lambda c}}e^{-\lambda x}$.
\end{enumerate}
\end{homeworkProblem}
\newpage
\begin{homeworkProblem}[4]
\solution
\begin{enumerate}[(a)]
    \item
    Because $M=max(U_1,U_2,U_3)$, then $P(M<=m)=P(U_1<=m,U_2<=m,U_3<=m)$.\\
    Because $U_1,U_2,U_3$ are i.i.d. Unif(0,1), then 
    \begin{align*}
        F_M(m)=P(M<=m)=P(U_1<=m)P(U_2<=m)P(U_3<=m)=m^3
    \end{align*}
    For, $0<=m<=1$.\\
    Then 
    \begin{align*}
        f_M(m)=3m^2
    \end{align*}
    For, $0<=m<=1$.\\
    Because $P(M<=m)=P(L<=l,M<=m)+P(L>l,M<=m)$,\\
    then $P(L<=l,M<=m)=P(M<=m)-P(L>l,M<=m)$.\\
    So we need to calculate $P(L>l,M<=m)$ first and we also have the confine of l, which is $0<=l<=m<=1$.\\
    \begin{align*}
        P(L>l,M<=m)&=P(min(U_1,U_2,U_3)>l,max(U_1,U_2,U_3)<=m)\\
        &=P(l<U_1<=m,l<U_2<=m,l<U_3<=m)\\
        &=(m-l)^3\\
    \end{align*}
    Then 
    \begin{align*}
        F_{L,M}(l,m)=(L<=l,M<=m)&=m^3-(m-l)^3\\
        &=3m^2l-3ml^2+l^3\\
    \end{align*}
    
    And then we can also get the PDF by respectively differentiating l and m, that is
    \begin{align*}
        f_{L,M}(l,m)=6(m-l)
    \end{align*}
    for $0<=l<=m<=1$.
\end{enumerate}
\begin{enumerate}[(b)]
    \item
    What we want to get is $f_{M|L}(m,l)=\frac{f_{M,L}(m,l)}{f_L(l)}$,\\
    so we just need to calculate $f_L(l)$, for $0<=l<=1$.\\
    \begin{align*}
        1-F_L(l)&=P(L>l)=P(U_1>l)P(U_2>l)P(U_3>l)=(1-l)^3\\
        F_L(l)&=1-(1-l)^3\\
        F_L(l)&=3l-3l^2+l^3\\
        f_L(l)&=3-6l+3l^2\\
        f_L(l)&=3(1-l)^2
    \end{align*}
    So $f_{M|L}(m,l)=\frac{2(m-l)}{(1-l)^2}$, for $0<=l<=m<=1$.\\
\end{enumerate}     
\end{homeworkProblem}
\newpage
\begin{homeworkProblem}[5]
\solution
\begin{enumerate}[(a)]
    \item 
    First we can divide this problem into three cases.\\
    case1, $l>m$, then $P(L=l,M=m)=0$\\
    case2, $l=m$, then $P(L=l,M=m)=P(X=l,Y=l)$\\
    Because they are independent Geom distribution, \\
    so $P(X=l,Y=l)=P(X=l)P(Y=l)=p*q^l*p*q^l=p^2*q^{2l}$.\\
    case3 is as follow,
    \begin{align*}
        P(L=l,M=m)&=P(X=l,Y=m)+P(X=m,Y=l)\\
        Because\ X,Y\ are\ i.i.d.\\
                  &=P(X=l)P(Y=m)+P(X=m)P(Y=l)\\
                  &=p*q^l*p*q^m+p*q^m*p*q^l\\
                  &=2p^2q^{m+l}
    \end{align*}
    From the cases we divied, obviously they are not independent. Because if we have known something about L, then we can also have some information about M, due to $L<=M$. 
\end{enumerate}
\begin{enumerate}[(b)]
    \item
    From (1), we can have
    \begin{align*}
    P(L=l)&=\sum_{m=0}^{\infty}P(L=l,M=m)\\
          &=P(L=l,M=l)+\sum_{m=l+1}^{\infty}P(L=l,M=m)\\
          &=p^2q^{2l}+2p^2q^l\sum_m{m=l+1}^{\infty}q^m\\
          &=p^2q^{2l}+2pq^{2l+1}
   \end{align*}
   By using story, all we need to do is to translate $L=min(X,Y)$.\\
    L can be seemed as X,Y at least happen one for the first success when L=l. So L is also a Geom distribution and $p_l=2p-p^2$.\\
    That is $L\sim Geom(2p-p^2)$.\\
    Then $P(L=l)=(2p-p^2)(1-2p+p^2)^l=(2p-p^2)((1-p)^2)^l=(2p-p^2)q^{2l}=(p^2+2p(1-p))q^{2l}=p^2q^{2l}+2pq^{2l+1}$\\
\end{enumerate}
\begin{enumerate}[(c)]
    \item
    Because $E[M+L]=E[X+Y]$, then we have
    \begin{align*}
        E[M]+E[L]&=E[X]+E[Y]\\
                E[M]&=E[X]+E[Y]-E[L]\\
                &=\frac{1-p}{p}+\frac{1-p}{p}-\frac{1-2p+p^2}{2p-p^2}\\
                &=\frac{(1-p)(3-p)}{(2-p)p}
    \end{align*}
\end{enumerate}
\begin{enumerate}[(d)]
    \item
    From the question and (a), and because $n>=0$, then we have
   \begin{align*}
    P(L=l,M-L=n)&=P(L=l,M=n+l)\\
                &=2p^2q^{l+n+l}\\
                &=2p^2q^{n+2l}
   \end{align*}
   So for the joint PMF of L and M-L, and because n and l are nonnegative integers, then we can use the theorem and get\\
   $f_{L,M-L}(l,n)=g(l)h(n)$, and $g(l)=a*q^{2l}$, and $h(n)=b*q^n$.\\
   To get the valid PMF, $g(l)=a\sum_{l=0}^{\infty}(q^2)^l=a*\frac{1}{1-q^2}=1$, then we can get $a=1-q^2$.\\
   And for another one, this must be valid too, so $h(n)=\frac{2p^2q^n}{1-q^2}$.\\
   Therefore $L,M-L$ are independent.
\end{enumerate}
\end{homeworkProblem}    
\end{document}
