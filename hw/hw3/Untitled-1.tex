\documentclass{article}

\usepackage{fancyhdr}
\usepackage{extramarks}
\usepackage{amsmath}
\usepackage{amsthm}
\usepackage{amsfonts}
\usepackage{tikz}
\usepackage[plain]{algorithm}
\usepackage{algpseudocode}
\usepackage{enumerate}
\usepackage{tikz}
\usepackage{xifthen}
\usepackage{xparse}
\usepackage{amsmath, amssymb}
\usepackage{lipsum}
\usetikzlibrary{automata,positioning}

%
% Basic Document Settings
%  

\topmargin=-0.45in
\evensidemargin=0in
\oddsidemargin=0in
\textwidth=6.5in
\textheight=9.0in
\headsep=0.25in

\linespread{1.1}

\pagestyle{fancy}
\lhead{\hmwkAuthorName}
\chead{\hmwkClass : \hmwkTitle}
\rhead{\firstxmark}
\lfoot{\lastxmark}
\cfoot{\thepage}

\renewcommand\headrulewidth{0.4pt}
\renewcommand\footrulewidth{0.4pt}

\setlength\parindent{0pt}

%
% Create Problem Sections
%

\newcommand{\enterProblemHeader}[1]{
    \nobreak\extramarks{}{Problem \arabic{#1} continued on next page\ldots}\nobreak{}
    \nobreak\extramarks{Problem \arabic{#1} (continued)}{Problem \arabic{#1} continued on next page\ldots}\nobreak{}
}

\newcommand{\exitProblemHeader}[1]{
    \nobreak\extramarks{Problem \arabic{#1} (continued)}{Problem \arabic{#1} continued on next page\ldots}\nobreak{}
    \stepcounter{#1}
    \nobreak\extramarks{Problem \arabic{#1}}{}\nobreak{}
}

\newcommand*\circled[1]{\tikz[baseline=(char.base)]{
		\node[shape=circle,draw,inner sep=2pt] (char) {#1};}}


\setcounter{secnumdepth}{0}
\newcounter{partCounter}
\newcounter{homeworkProblemCounter}
\setcounter{homeworkProblemCounter}{1}
\nobreak\extramarks{Problem \arabic{homeworkProblemCounter}}{}\nobreak{}

%
% Homework Problem Environment
%
% This environment takes an optional argument. When given, it will adjust the
% problem counter. This is useful for when the problems given for your
% assignment aren't sequential. See the last 3 problems of this template for an
% example.
%

\newenvironment{homeworkProblem}[1][-1]{
    \ifnum#1>0
        \setcounter{homeworkProblemCounter}{#1}
    \fi
    \section{Problem \arabic{homeworkProblemCounter}}
    \setcounter{partCounter}{1}
    \enterProblemHeader{homeworkProblemCounter}
}{
    \exitProblemHeader{homeworkProblemCounter}
}

%
% Homework Details
%   - Title
%   - Class
%   - Due date
%   - Name
%   - Student ID

\newcommand{\hmwkTitle}{Homework\ \#03}
\newcommand{\hmwkClass}{Probability \& Statistics for EECS}
\newcommand{\hmwkDueDate}{March 05, 2023}
\newcommand{\hmwkAuthorName}{Wang Yunfei}
\newcommand{\hmwkAuthorID}{2021533135}


%
% Title Page
%

\title{
    \vspace{2in}
    \textmd{\textbf{\hmwkClass:\\  \hmwkTitle}}\\
    \normalsize\vspace{0.1in}\small{Due\ on\ \hmwkDueDate\ at 23:59}\\
	\vspace{4in}
}

\author{
	Name: \textbf{\hmwkAuthorName} \\
	Student ID: \hmwkAuthorID}
\date{}

\renewcommand{\part}[1]{\textbf{\large Part \Alph{partCounter}}\stepcounter{partCounter}\\}

%
% Various Helper Commands
%

% Useful for algorithms
\newcommand{\alg}[1]{\textsc{\bfseries \footnotesize #1}}
% For derivatives
\newcommand{\deriv}[1]{\frac{\mathrm{d}}{\mathrm{d}x} (#1)}
% For partial derivatives
\newcommand{\pderiv}[2]{\frac{\partial}{\partial #1} (#2)}
% Integral dx
\newcommand{\dx}{\mathrm{d}x}
% Alias for the Solution section header
\newcommand{\solution}{\textbf{\large Solution}}
% Probability commands: Expectation, Variance, Covariance, Bias
\newcommand{\E}{\mathrm{E}}
\newcommand{\Var}{\mathrm{Var}}
\newcommand{\Cov}{\mathrm{Cov}}
\newcommand{\Bias}{\mathrm{Bias}}

\begin{document}

\maketitle

\pagebreak

\begin{homeworkProblem}[1]
    \solution
        \begin{enumerate}
            \item We can directly calculate the probability of the system working and divide it into the following 5 cases.\\
            First, five devices all function and there is just one situation, that is $P(5)=1*p^5$.\\
            Second, four devices function and there are $\binom{5}{4}=5$ situations, that is $P(4)=5*p^4*(1-p)$.\\ 
            Third, two devices funtion and there are two situations, those are 1 with 3 and 2 with 4. So $P(2)=2*p^2*(1-p)^3$.\\
            Fourth, there devices function and there are eight situations. On the basis of 1 and 3 we have 3 cases, on the basis of 2 and 4 we also have 3 cases and on the crossover we have 2 cases namely 1, 4, 5 and 2, 3, 5.\\
            Therefore, $P(3)=8*p^3*(1-p)^2$.\\
            Fifth, one device cannot let system function, and it is similar with the condition of zero device.\\
            Finally, $P(system \ functions)=P(2)+P(3)+P(4)+P(5)=2p^5-5p^4+2p^3+2p^2$.
        \end{enumerate}
\end{homeworkProblem}

\begin{homeworkProblem}[2]
\solution
\begin{enumerate}[(a)]
    \item Assume event C: admit an applicant.\\
    Assume event A: applicant is good at Apex.\\
    Assume event B: applicant is good at Genshin Impact.\\
    And from the question, we have $C = A \ or \  B$.\\
    Then we have $1=P(A\cup B|C )=P(A|C)+P(B|C)-P(A\cap B|C )$. And then we can assume two events are disjoint, then we can explain it intuitively, because the sum of the probability of two events is 1.
    Due to the increasement of $P(A)$, $P(B)$ must decrease. And we can also intuitively explain similarly when two events are joint, because we donnot need to care the part where they are joint.
\end{enumerate}
\begin{enumerate}[(b)]
    \item From the question and the Bayes' formula, we have $P(A\cap B\cap C)=P(A\cap B|C)P(C)=P(A|B\cap C)P(B|C)P(C)$.\\
    and then we can eliminate $P(C)$ at both sides, and then have $P(A\cap B|C)=P(A|B\cap C)P(B|C)$.\\
    And we have $P(A | B \cap C) < P(A | C)$. Therefore we can prove $P(A\cap B|C)\neq P(B|C)P(A |C)$.\\
    So A and B are conditionally dependent given C.
  
\end{enumerate}
\end{homeworkProblem}

\begin{homeworkProblem}[3]
\solution
\begin{enumerate}[(a)]
    \item $p_n=\frac{1}{6}(p_{n-1}+p_{n-2}+p_{n-3}+p_{n-4}+p_{n-5}+p_{n-6})$ for $n>0$. At the meantime, $p_0=1$ and $p_k=0$ for $k<0$.
\end{enumerate}
\begin{enumerate}[(b)]
    \item 
    First, we need to calculate $p_1$, $p_2$, $p_3$, $p_4$, $p_5$, $p_6$. \\
    $p_1=\frac{1}{6}(p_{-1}+p_{-2}+p_{-3}+p_{-4}+p_{-5}+p_{0})=\frac{1}{6}$.\\
    $p_2=\frac{1}{6}(p_{1}+p_{-1}+p_{-2}+p_{-3}+p_{-4}+p_{0})=\frac{7}{36}$.\\
    $p_3=\frac{1}{6}(p_{1}+p_{2}+p_{-1}+p_{-2}+p_{-3}+p_{0})=\frac{49}{216}$.\\
    $p_4=\frac{1}{6}(p_{1}+p_{2}+p_{3}+p_{-1}+p_{-2}+p_{0})=\frac{343}{1296}$.\\
    $p_5=\frac{1}{6}(p_{1}+p_{2}+p_{3}+p_{4}+p_{-1}+p_{0})=\frac{2401}{7776}$.\\
    $p_6=\frac{1}{6}(p_{1}+p_{2}+p_{3}+p_{4}+p_{5}+p_{0})=0.3602$.\\
    $p_7=\frac{1}{6}(p_{1}+p_{2}+p_{3}+p_{4}+p_{5}+p_{6})=0.2536$.\\
\end{enumerate}
\begin{enumerate}[(c)]
    \item The expectation for each throw of the dice is $E[one \ throw\ of \ dice]=\frac{1+2+3+4+5+6}{6}=\frac{7}{2}$. \\
    And from the definition of the expectation we can interpret this as meaning the roll will hit 1 in every 3.5 times. \\
    Therefore, the probability of $p_n=\frac{1}{3.5}=\frac{2}{7}$ when n converging to infinity.
  
\end{enumerate}
\end{homeworkProblem}
\begin{homeworkProblem}[4]
\solution
    \begin{enumerate}[(a)]
        \item 
        $p_{i,j}=\frac{n-j+1}{n}p_{i-1,j-1}+\frac{j}{n}p_{i-1,j}$ for $i>=2$ and $1<=j<=n$.\\
        Also for the purpose of explanation in b we define other cases here:\\
        First, $p_{i,j}=0$ for $j>i$.\\
        Second, $p_{i,j}=0$ for $j=0$ or $i=0$.\\
        Third, $p_{1,1}=1$.\\
    \end{enumerate}
    \begin{enumerate}[(b)]
        \item From the recursive equation and the definition, we can calculate $p_{i,j}$ by considering $p_{i,j}$ as a matrix, in which i is the column and j is row.\\
        At the beginning, we can calculate $p_{i,1}$ firstly for $i>=2$, because $p_{i,0}=0$. \\
        Therefore, we can use $p_{1,1}$ to column-by-column calculate $p_{i,1}$ for $i>=2$ by recursive equation.\\
        Secondly, we can use $p_{i,1}$ for $i>=2$ which we have calculated and $p_{1,2}=0$ to column-by-column calculate $p_{i,2}$ for $i>=2$ by recursive equation.\\
        $\cdots$\\
        Finally, we can step by step to calculate the jth row, and get $p_{i,j}$ similarly for $i>=2$.\\
        So we can get it by calculating layer-by-layer.
      
    \end{enumerate}
\end{homeworkProblem}
\begin{homeworkProblem}[5]
\solution
    \begin{enumerate}[(a)]
        \item $p_k=p*p_{k-1}+q*p_{k+1}$ for all $k>=0$. And for $p_0$, because we must start at the original point, then $p_0=1$.
      
    \end{enumerate}
    \begin{enumerate}[(b)]
        \item First we can simplify the recursive equation to obtain the isoperimetric series, that is\\
        $\frac{p}{q}=\frac{p_{k+1}-p_k}{p_k-p_{k-1}}$, and we can assume $\alpha_k=\frac{p}{q}\alpha_{k-1}$.\\
        We can express $p_k$ by summing the equiprobable series as $p_k-p_0=\frac{(p_1-p_0)*(1-(\frac{p}{q})^k)}{1-\frac{p}{q}}$.\\
        So what we need to get is just $p_1$.\\
        Because if we finally reach 1 and assume one process has been finished, and then we start another process to reach k. Therefore through this way we can also get $p_k$, by considering the new original point as 1 not 0.
        So we have a new formula $p_k=p_1*p_{k-1}$. And then we can have $p_k=p_1^k$.\\
        And then we can calculate $p_1$, through the formula and the recursive equation, and we have $p_1=p_0*p+p_2*q=p+p_1^2q$.\\
        And then we have one equation about $p_1$, that is $p_1^2q-p_1+p$. And then we can use the root formula $p_{1,1or2}=\frac{1\pm \sqrt[2]{1-4pq}}{2q}$.\\
        And then we can simplify it as $p_{1,1or2}=\frac{1\pm \left\lvert 2p-1\right\rvert }{2q}$.\\
        For $p=0.5$, $p_{1,1or2}=1$, and then $p_1=1$. So $p_k=1$.\\
        For $p>0.5$, and because $0<=p_1<=1$, we have $p_1=1$. So $p_k=1$.\\
        For $p<0.5$, and because $0<=p_1<=1$, we have $p_1=\frac{p}{q}$ or $p_1=1$. So when $p_1=1$, $p_k=1$. And for another case, $p_k=(\frac{p}{q})^k$. \\

      
    \end{enumerate}
\end{homeworkProblem}    
\end{document}
