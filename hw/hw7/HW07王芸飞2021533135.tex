\documentclass{article}

\usepackage{fancyhdr}
\usepackage{extramarks}
\usepackage{amsmath}
\usepackage{amsthm}
\usepackage{amsfonts}
\usepackage{tikz}
\usepackage[plain]{algorithm}
\usepackage{algpseudocode}
\usepackage{enumerate}
\usepackage{tikz}
\usepackage{xifthen}
\usepackage{xparse}
\usepackage{amsmath, amssymb}
\usepackage{lipsum}
\usetikzlibrary{automata,positioning}

%
% Basic Document Settings
%  

\topmargin=-0.45in
\evensidemargin=0in
\oddsidemargin=0in
\textwidth=6.5in
\textheight=9.0in
\headsep=0.25in

\linespread{1.1}

\pagestyle{fancy}
\lhead{\hmwkAuthorName}
\chead{\hmwkClass : \hmwkTitle}
\rhead{\firstxmark}
\lfoot{\lastxmark}
\cfoot{\thepage}

\renewcommand\headrulewidth{0.4pt}
\renewcommand\footrulewidth{0.4pt}

\setlength\parindent{0pt}

%
% Create Problem Sections
%

\newcommand{\enterProblemHeader}[1]{
    \nobreak\extramarks{}{Problem \arabic{#1} continued on next page\ldots}\nobreak{}
    \nobreak\extramarks{Problem \arabic{#1} (continued)}{Problem \arabic{#1} continued on next page\ldots}\nobreak{}
}

\newcommand{\exitProblemHeader}[1]{
    \nobreak\extramarks{Problem \arabic{#1} (continued)}{Problem \arabic{#1} continued on next page\ldots}\nobreak{}
    \stepcounter{#1}
    \nobreak\extramarks{Problem \arabic{#1}}{}\nobreak{}
}

\newcommand*\circled[1]{\tikz[baseline=(char.base)]{
		\node[shape=circle,draw,inner sep=2pt] (char) {#1};}}


\setcounter{secnumdepth}{0}
\newcounter{partCounter}
\newcounter{homeworkProblemCounter}
\setcounter{homeworkProblemCounter}{1}
\nobreak\extramarks{Problem \arabic{homeworkProblemCounter}}{}\nobreak{}

%
% Homework Problem Environment
%
% This environment takes an optional argument. When given, it will adjust the
% problem counter. This is useful for when the problems given for your
% assignment aren't sequential. See the last 3 problems of this template for an
% example.
%

\newenvironment{homeworkProblem}[1][-1]{
    \ifnum#1>0
        \setcounter{homeworkProblemCounter}{#1}
    \fi
    \section{Problem \arabic{homeworkProblemCounter}}
    \setcounter{partCounter}{1}
    \enterProblemHeader{homeworkProblemCounter}
}{
    \exitProblemHeader{homeworkProblemCounter}
}

%
% Homework Details
%   - Title
%   - Class
%   - Due date
%   - Name
%   - Student ID

\newcommand{\hmwkTitle}{Homework\ \#07}
\newcommand{\hmwkClass}{Probability \& Statistics for EECS}
\newcommand{\hmwkDueDate}{April 02, 2023}
\newcommand{\hmwkAuthorName}{Wang Yunfei}
\newcommand{\hmwkAuthorID}{2021533135}


%
% Title Page
%

\title{
    \vspace{2in}
    \textmd{\textbf{\hmwkClass:\\  \hmwkTitle}}\\
    \normalsize\vspace{0.1in}\small{Due\ on\ \hmwkDueDate\ at 23:59}\\
	\vspace{4in}
}

\author{
	Name: \textbf{\hmwkAuthorName} \\
	Student ID: \hmwkAuthorID}
\date{}

\renewcommand{\part}[1]{\textbf{\large Part \Alph{partCounter}}\stepcounter{partCounter}\\}

%
% Various Helper Commands
%

% Useful for algorithms
\newcommand{\alg}[1]{\textsc{\bfseries \footnotesize #1}}
% For derivatives
\newcommand{\deriv}[1]{\frac{\mathrm{d}}{\mathrm{d}x} (#1)}
% For partial derivatives
\newcommand{\pderiv}[2]{\frac{\partial}{\partial #1} (#2)}
% Integral dx
\newcommand{\dx}{\mathrm{d}x}
% Alias for the Solution section header
\newcommand{\solution}{\textbf{\large Solution}}
% Probability commands: Expectation, Variance, Covariance, Bias
\newcommand{\E}{\mathrm{E}}
\newcommand{\Var}{\mathrm{Var}}
\newcommand{\Cov}{\mathrm{Cov}}
\newcommand{\Bias}{\mathrm{Bias}}

\begin{document}

\maketitle

\pagebreak

\begin{homeworkProblem}[1]
\solution
\begin{enumerate}[(a)]
    \item
    First check whether it is valid:\\
    (1)increasing:$F'(x)=\frac{1}{\pi\sqrt[2]{x}\sqrt[2]{1-x}}$, and $1<x<0$, so $F'(x)>0$. Therefore, F is monotone increasing.\\
    (2)right-continuous: Because $argsinx$ is continuous when $0<x<1$ without doubts, then so do $F(x)$.\\
    (3)limit: when x is $0^+$, $F(x)\approx\frac{2}{\pi}x$, then $F(0^+)=0$.\\
    and when x is $1^-$, $F(x)=\frac{2}{\pi}\frac{\pi}{2}=1$. Therefore it satisfies the limitation.\\
    And then $f(x)=\frac{1}{\pi\sqrt[2]{x}\sqrt[2]{1-x}}$ for $0<x<1$. Otherwise 0.\\
\end{enumerate}
\begin{enumerate}[(b)]
    \item
    That is because PDF means Probability desity function, and when we want to calculate the probability of any concrete point, it is equal to 0 anyway. 
    Therefore, even though f(x) goes to $\infty$ as x approaches 0 or 1, it doesnot matter, because it is just one point. It is still possible for us to calculate the area of the PDF and let it be equal to 1.
\end{enumerate}
\end{homeworkProblem}

\begin{homeworkProblem}[2]

\begin{enumerate}[a]
    \item
    Because F is a CDF which is continuous and strictly increasing, then we can let $U\sim Unif(0,1)$, and $X=F^{-1}(U)$.\\
    And the meantime, $F(X)=F(F^{-1}(U))=U$, so we can have $u=F(x)$.\\
    And what we want is the area under the curve of the quantile function from 0 to 1.\\
    So $\int_{0}^{1} F^{-1}(u) \,du =\int_{-\infty}^{\infty} F^{-1}(F(x)) \,dF(x) =\int_{-\infty}^{\infty} xf(x) \,dx=\mu $.\\
    Therefore we get it.
\end{enumerate}

\end{homeworkProblem}

\begin{homeworkProblem}[3]
    \solution
    \begin{enumerate}[(a)]
        \item
        First for $U_i$, $1<=i<=n$, they are i.i.d. $Unif(0,1)$, then we can have:
        \begin{align*}
            P(X<x)&=P(X<=x)\\
                  &=P(max(U_1,\cdots,U_n)<=x)\\
                  &=P(U_1<=x,\cdots,U_n<=x)\\
        Because\ they\ are\ independent\\
                  &=P(U_1<=x)\cdots P(U_n<=x)\\
                  &=\frac{x-0}{1}\cdots \frac{x-0}{1}\\
                  &=x^n\\
        \end{align*}
        So because we have got $CDF$ of $X$, then $PDF$ of $X$ is $nx^{n-1}$, when $0<x<1$. Otherwise is 0.\\
        At the same time 
        \begin{align*}
            E[X]&=\int_{0}^{1}x*n*x^{n-1}  \,dx\\
                &=n\int_{0}^{1}x^{n}  \,dx\\
                &=\frac{n}{n+1}\\
        \end{align*}
    \end{enumerate}
\end{homeworkProblem}
\begin{homeworkProblem}[4]
    \solution
        \begin{enumerate}[(a)]
            \item
            Obviously, $X+Y=1$, so we can have $R=\frac{X}{Y}=\frac{X}{1-X}$. And we also have the support of $R$ is $(0,1)$\\
            Then for $F(R)=P(R<=x)=P(\frac{X}{1-X}<=x)=P(X<=\frac{x}{1+x})$, and $0<=x<=1$.\\
            And for X, we have $P(X<=x)=2x$ for $0<=x<=0.5$.\\
            Therefore, we can have $P(R<=x)=\frac{2x}{1+x}$, for $0<=x<=1$.\\
            And then, we can also get PDF of R, that is $f(R)=F'(R)=\frac{2}{(1+x)^2}$, for $0<=x<=1$.\\
        \end{enumerate}
        \begin{enumerate}[(b)]
            \item
          $E[R]=\int_{0}^{1}x\frac{2}{(1+x)^2}  \,dx =2ln2-1$.
        \end{enumerate}
        \begin{enumerate}[(c)]
            \item
          $E[\frac{1}{R}]=\int_{0}^{1}\frac{2}{x(1+x)^2}  \,dx =\infty$.
        \end{enumerate}
    \end{homeworkProblem}
\begin{homeworkProblem}[5]
\solution
\begin{enumerate}[(a)]
            \item
            $T=G\varDelta t$.
\end{enumerate}
\begin{enumerate}[(b)]
    \item
    \begin{align*}
        P(T>=t)&=P(G\varDelta t>=t)\\
                &=P(G>=\frac{t}{\varDelta t})\\
                &=P(G>=\left\lfloor \frac{t}{\varDelta t}\right\rfloor)\\
                &=(1-\lambda \varDelta t)^{\left\lfloor \frac{t}{\varDelta t}\right\rfloor}\\
    \end{align*}
    Therefore, the CDF of T is $1-(1-\lambda \varDelta t)^{\left\lfloor \frac{t}{\varDelta t}\right\rfloor}$.\\
\end{enumerate}
\begin{enumerate}[(c)]
    \item 
        \begin{align*}
            1-(1-\lambda \varDelta t)^{\left\lfloor \frac{t}{\varDelta t}\right\rfloor}&=\lim_{ \varDelta t\to 0} 1-(1-\lambda \varDelta t)^{\left\lfloor \frac{t}{\varDelta t}\right\rfloor}\\
                                                                                       &=1-\lim_{ \varDelta t\to 0}(1-\lambda \varDelta t)^{ \frac{t}{\varDelta t}}\\
                                                                                       &=1-\lim_{ \varDelta t\to 0} (1-\lambda \varDelta t)^{ \frac{\lambda t}{\lambda \varDelta t}}\\
                                                                                       &=1-(\frac{1}{e})^{\lambda t}\\
                                                                                       &=1-e^{-\lambda t}\\
        \end{align*}
       Therefore, the CDF of T converges to the Expo($\lambda$) CDF. 
\end{enumerate}
\end{homeworkProblem}    
\begin{homeworkProblem}[6]
        \solution
        \begin{enumerate}[(a)]
            \item
            For $E[max(Z-c,0)]$,
            \begin{align*}
                E[max(Z-c,0)]&=\int_{-\infty}^{c}0  \,dz +\int_{c}^{\infty}(z-c)\varphi(z)  \,dz \\
                             &=0+\int_{-\infty}^{-c}-z\varphi(z)  \,dz-c\int_{c}^{\infty}\varphi(z)  \,dz \\
                             &=\varphi(-c)-\varphi(-\infty)-c(1-\Phi(c))\\
                Because\ of\ the\ symmetry\ of\ PDF\\
                             &=\varphi(c)-0-c(1-\Phi(c))\\
                             &=\varphi(c)-c(1-\Phi(c))\\
            \end{align*}
        \end{enumerate}
    \end{homeworkProblem}
\end{document}
